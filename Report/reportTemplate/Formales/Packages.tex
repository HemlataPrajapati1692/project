% Anpassung des Seitenlayouts --------------------------------------------------
%   siehe Seitenstil.tex
% ------------------------------------------------------------------------------
%\usepackage[
%    automark, % Kapitelangaben in Kopfzeile automatisch erstellen
%    headsepline, % Trennlinie unter Kopfzeile
%    ilines % Trennlinie linksbündig ausrichten
%]{scrpage2}
%
% Nutzung des aktuelleren Pakets
\usepackage[
    automark, % Kapitelangaben in Kopfzeile automatisch erstellen
    headsepline, % Trennlinie unter Kopfzeile
    ilines % Trennlinie linksbündig ausrichten
]{scrlayer-scrpage}
% zur Vermeidung von float-warnings
\usepackage{scrhack}
% für bessere Umbruchpunkte
% siehe: https://texfragen.de/overfull_hbox
\usepackage{microtype}

% Moderneres Backend für das Literaturverzeichnis
% Quelle: https://golatex.de/vorlage-fuer-abschlussarbeiten-htwk-leipzig-t18498.html
%
\usepackage[
    backend=biber,
%    style=numeric-comp,  % entspricht dem Stil der AMS
%    style=authoryear, % entspricht dem Harvard-Stil
    style=ieee, % entspricht dem IEEE-Stil
%    natbib,
    sorting=nyt,
    defernumbers=true,
    backref=true,
    giveninits=true, % Initialen statt vollständige Vornamen
    uniquename=init,
    doi=false,
    isbn=false
]{biblatex}
% Familienname in Käpitälchen im Literaturverzeichnis
\renewcommand*{\mkbibnamefamily}[1]{\textsc{#1}}
%\renewcommand*{\mkbibnamefirst}[1]{#1\addcomma}
%\renewcommand*{\mkbibnameprefix}[1]{\textsc{#1}}
\addbibresource{Bibtex/sources.bib}
\setlength{\bibitemsep}{0.5\baselineskip}
% siehe: https://tex.stackexchange.com/questions/39285/whats-the-advantage-of-using-csquotes-over-using-an-editors-auto-replacement-f
\usepackage[babel, german=quotes]{csquotes}
\renewcommand{\mkcitation}[1]{#1}

% Schrift ----------------------------------------------------------------------
\usepackage{lmodern} % bessere Fonts
\usepackage{relsize} % Schriftgröße relativ festlegen

% Grafiken ---------------------------------------------------------------------
% Einbinden von JPG-Grafiken ermöglichen
\usepackage[dvips,final]{graphicx}
% hier liegen die Bilder des Dokuments
\graphicspath{{Bilder/}}

% Befehle aus AMSTeX für mathematische Symbole z.B. \boldsymbol \mathbb --------
\usepackage{amsmath,amsfonts}

% für Index-Ausgabe mit \printindex --------------------------------------------
\usepackage{makeidx}

% Einfache Definition der Zeilenabstände und Seitenränder etc. -----------------
\usepackage{setspace}
\usepackage{geometry}

% zum Umfließen von Bildern ----------------------------------------------------
\usepackage{floatflt}
% Zum Erzwingen der Platzierung von Tabellen und Abbildungen mitels [H] --------
\usepackage{float}

\usepackage[dvipsnames]{xcolor}        %  erm"oglicht Benutzung von Farbe durch Namensgebung 

% fortlaufendes Durchnummerieren der Fußnoten ----------------------------------
\usepackage{chngcntr}

% für lange Tabellen -----------------------------------------------------------
\usepackage{booktabs}
% für lange Tabellen -----------------------------------------------------------
\usepackage{longtable}
\usepackage{array}
\usepackage{ragged2e}
\usepackage{lscape}

% Spaltendefinition rechtsbündig mit definierter Breite ------------------------
\newcolumntype{w}[1]{>{\raggedleft\hspace{0pt}}p{#1}}

% Formatierung von Listen ändern -----------------------------------------------
\usepackage{paralist}

% bei der Definition eigener Befehle benötigt
\usepackage{ifthen}

% definiert u.a. die Befehle \todo, \missingfigure und \listoftodos
% siehe: https://mirror.hmc.edu/ctan/macros/latex/contrib/todonotes/todonotes.pdf
\usepackage{todonotes}

% sorgt dafür, dass Leerzeichen hinter parameterlosen Makros nicht als Makroendezeichen interpretiert werden
\usepackage{xspace}

% stellt zusätzliche Symbole bereit (z.B. \Square == Checkbox)
% siehe: http://texdoc.net/texmf-dist/doc/latex/wasysym/wasysym.pdf
\usepackage{wasysym}

% flexible Verweise
% siehe: https://ctan.org/pkg/varioref
\usepackage{varioref}

% zum Einbinden von Programmcode -----------------------------------------------
\usepackage{listings}
%
\definecolor{hellgrau}{rgb}{0.95,0.95,0.95}
\definecolor{dkgreen}{rgb}{0,0.6,0}
\definecolor{hellgruen}{rgb}{.7,1,.7}
\definecolor{gray}{rgb}{0.5,0.5,0.5}
\definecolor{mauve}{rgb}{0.58,0,0.82}
\definecolor{orange1}{RGB}{255,153,102}
\definecolor{hellgelb}{RGB}{255,255,204}
\definecolor{hellgrau1}{RGB}{204,204,204}
\definecolor{LightGray}{gray}{0.9} % new color for background of listing
%
\lstset{ %
  language=Python,                % the language of the code
  %basicstyle=\small, % print whole listing small
  %basicstyle=\footnotesize,       % the size of the fonts that are used for the code
  basicstyle=\ttfamily\footnotesize,
  numbers=left,                   % where to put the line-numbers
  numberstyle=\tiny\color{gray},  % the style that is used for the line-numbers
  stepnumber=1,                   % the step between two line-numbers. If it's 1, each line 
                                  % will be numbered
  numbersep=5pt,                  % how far the line-numbers are from the code
  backgroundcolor=\color{LightGray},  % choose the background color. You must add \usepackage{color}
  showspaces=false,               % show spaces adding particular underscores
  showstringspaces=false,         % underline spaces within strings
  showtabs=false,                 % show tabs within strings adding particular underscores
  frame=single,                   % adds a frame around the code
  rulecolor=\color{black},        % if not set, the frame-color may be changed on line-breaks within not-black text (e.g. commens (green here))
  tabsize=4,                      % sets default tabsize to 4 spaces (in accordance with Python)
  captionpos=b,                   % sets the caption-position to bottom
  breaklines=true,                % sets automatic line breaking
  breakatwhitespace=false,        % sets if automatic breaks should only happen at whitespace
  title=\lstname,                 % show the filename of files included with \lstinputlisting;
                                  % also try caption instead of title
  keywordstyle=\color{blue},      % keyword style
  commentstyle=\color{dkgreen},   % comment style
  stringstyle=\color{mauve},      % string literal style
  escapeinside={\%*}{*)},         % if you want to add LaTeX within your code
  morekeywords={*,...}            % if you want to add more keywords to the set
}

% Umlaute in Listings (Kommentaren) verarbeiten
\lstset{
  literate={ö}{{\"o}}1
           {ä}{{\"a}}1
           {ü}{{\"u}}1
           {Ö}{{\"O}}1
           {Ä}{{\"A}}1
           {Ü}{{\"U}}1
           {ß}{{\"s}}1
}

% URL verlinken, lange URLs umbrechen etc. -------------------------------------
\usepackage{url}

% wichtig für korrekte Zitierweise ---------------------------------------------
% ersetzt durch biblatex + biber
%\usepackage[square]{natbib}

% PDF-Optionen -----------------------------------------------------------------
%
\usepackage[
    bookmarks,
    bookmarksopen=true,
    colorlinks=true,
%    backref, -- nicht mit biblatex vereinbar; dortige Option verwenden!
    plainpages=false, % zur korrekten Erstellung der Bookmarks
    pdfpagelabels, % zur korrekten Erstellung der Bookmarks
    hypertexnames=true, % zur korrekten Erstellung der Bookmarks
    linktocpage % Seitenzahlen anstatt Text im Inhaltsverzeichnis verlinken
]{hyperref}
% 
% Nutzung des Schalters für Druck/PDF (Datei Abschlussarbeit.tex Zeilen 46 -- 48)
%
\ifprint 
% alles schwarz
\hypersetup{
    linkcolor=black,
    anchorcolor=black,% Ankertext
    citecolor=black, % Verweise auf Literaturverzeichniseinträge im Text
    filecolor=black, % Verknüpfungen, die lokale Dateien öffnen
    menucolor=black, % Acrobat-Menüpunkte
    urlcolor=black % Verweise auf Webseiten
    }
\else
% farbig
\hypersetup{
    linkcolor=CadetBlue,
    anchorcolor=black,% Ankertext
    citecolor=blue, % Verweise auf Literaturverzeichniseinträge im Text
    filecolor=magenta, % Verknüpfungen, die lokale Dateien öffnen
    menucolor=red, % Acrobat-Menüpunkte
    urlcolor=ForestGreen % Verweise auf Webseiten
	}
\fi
%
% Meta-Informationen
%
\hypersetup{
    pdftitle={\titel \untertitel},
    pdfauthor={\verfasser},
    pdfcreator={\verfasser},
    pdfsubject={\titel \untertitel},
    pdfkeywords={\titel \untertitel},
}

% ---- Paket für Abkürzungen, Symbole und Erklärungen ---
% Einführung siehe: https://www.dante.de/events/dante2015/Programm/vortraege/vortrage-partosch.pdf
\usepackage[%
toc, % ins Inhaltsverzeichnis übernehmen
nonumberlist, % keine Seitenzahlen
symbols, % ermöglichst Symbole+Symbolverzeichnis
acronym % ermoeglicht Abkuerzungen+Abkuerzungsverz.
]{glossaries} % ermoeglicht Glossar, ...
\renewcommand*{\glspostdescription}{}
%
% ---- Paket für Bilder
%
\usepackage{tikz}
\usetikzlibrary{positioning} % für Beispiel mit Listing

