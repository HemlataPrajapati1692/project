\begin{longtable}{p{.25\textwidth}p{.47\textwidth}p{.25\textwidth}}
\caption{Die 7 Design Science Research Guidelines in Anlehnung an \cite{Hevneretal2004}}
\label{tab:DSResearchGuidelines}
\toprule
\textbf{Guideline} & \textbf{Kurzbeschreibung} & \textbf{Erzielt durch} \\*
\midrule
1: Design as an Artifact & Design Science Research muss ein realisierbares Artefakt herstellen & Prototypische Implementierung  \\*
\midrule
2: Problem Relevance & Technologische Lösung zu einem relevanten, betrieblichen Problem & Experteninterviews \\*
\midrule
3: Design Evaluation & Mehrwert des Artefakts muss umfassend getestet werden & Validierungscheck mit
Dummydaten und Testdaten  \\*
\midrule
4: Research Contributions & Verifizierbare, eindeutige Beiträge zur Forschung müssen erzielt werden & Cloudbasierte Advanced Analytics Anwendung für effizientere Qualitätsanalysen \\*
\midrule
5: Research Rigor & Eindeutig nachvollziehbare Methoden zur
Erstellung und Evaluation des Artefakts & angewendete Forschungsmethoden  \\*
\midrule
6: Design as a Search Process & Suche nach einem effektiven Artefakt
erfordert Feedback und Methoden & Rücksprache mit Stakeholdern \\*
\midrule
7: Communication of Research & Design Science Research muss den Zielgruppen verständlich präsentiert werden &
Strukturierte und nachvollziehbare Dokumentation \\*
\bottomrule
\end{longtable}
