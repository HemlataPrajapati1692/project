\section{Abgabe der Arbeit}
\label{anh:Anh-Abgabe}
%
Die Regelungen zur Abgabe der Abschlussarbeit (neben der selbstverständlichen Termingerechtigkeit) sind die Folgenden:
%
\subsection{Exemplare}
\label{subsec:Anh-Abgabe-Exemplare}
%
\begin{compactenum}[(a)]
\item \emph{Druckexemplare}: 
  Hiervon sind 2 abzugeben. Dabei ist darauf zu achten, dass die Ausdrucke keinen farbigen Text für Verweise
  enthalten.
\item \emph{Elektronische Exemplare}: 
  Es ist jeweils ein elektronisches Exemplar der Abschlussarbeit als PDF auf den vereinbarten Cloud-Speicher hochzuladen und
  auch auf der dem Druckexemplar beigefügten CD oder DVD muss sich eines befinden. Im PDF müssen Verweise als solche
  erkennbar sein und interne Verweise sollten sich von Verweisen auf Quellen ebenso unterscheiden wie von Verweisen
  auf Internetressourcen.
\end{compactenum}
%
\subsection{Literatur}
\label{subsec:Anh-Abgabe-Literatur}
%
\begin{compactenum}[(a)]
\item \emph{Bib\TeX{}-Datei}:
  Die in der Arbeit verwendeten Quellen sind auch in Form einer \Software{BibTeX}-Datei elektronisch abzugeben. Wie
  die Arbeit selbst ist diese Datei in den Cloud-Speicher hochzuladen und auf den Datenträger zu brennen.
\item \emph{Downloads}:
  Sollten Literaturquellen kostenlos online verfügbar sein, so sind diese herunterzuladen und auf den einzureichenden
  Datenträger zu brennen.
\item \emph{Internetquellen}:
  Internetquellen sind in Offline lesbarer Form auf dem Datenträger beizufügen.
\end{compactenum}
%
\subsection{Quellcode}
\label{subsec:Anh-Abgabe-Quellcode}
%
Quellcode ist der Arbeit auf dem Datenträger beizufügen. Er ist insbesondere \underline{nicht} in Gänze auszudrucken!