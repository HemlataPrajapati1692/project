\chapter{Fazit und kritische Bewertung}
\label{cha:Fazit}
%
\begin{quote}
\glqq 
Da \LaTeX{} ein überholtes und auslaufendes System zur Formatierung
von Text darstellt, entschloss ich mich vor rund dreißig Jahren, stattdessen
nur das Textverarbeitungssystem eines namhaften Herstellers zu
verwenden. \emph{Vielleicht war das falsch.}\grqq
\end{quote}
%
Das obige Zitat aus der \LaTeX{}-Einführung von \textcite{Gitter2018} fasst bestens zusammen, was mit
diesem Dokument vermittelt werden sollte. Dabei 
enthält es Beispiele für einige relevante Themen für die Erstellung einer Abschlussarbeit, die
unabhängig vom verwendeten Textbearbeitungs- und Satzprogramm zu beachten sind. Darüberhinaus dient es als
Vorlage für die Erstellung von Abschlussarbeiten mit \LaTeXe{}. Hier werden die wesentlichen Techniken
exemplifiziert, so dass diese leicht von Erstellern einer Abschlussarbeit im eigenen Kontext verwendet
werden können.

Allerdings enthebt dies die Autoren nicht der Pflicht, sich weitere nötige Kenntnisse selbst anzueignen. Zu
diesem Zweck sind weitere \LaTeX{}-bezogene Dokumente beigefügt, die als Startpunkt dienen können. Ansonsten ist
natürlich das Internet eine schier unerschöpfliche Quelle des Wissens zum Thema \TeX{}/\LaTeX{}, insbesondere
Foren wie \url{https://tex.stackexchange.com} oder 
\url{https://www.mrunix.de/forums/forumdisplay.php?38-LaTeX-Forum} oder auch die Webseiten von
DANTE (\gls{dante}). Aber es existieren auch freie Bücher zum Thema, nicht zuletzt das umfassende Werk
auf \textcite{Wikibooks2018}.

Bei der Verwendung von \LaTeX{} entsteht ein Dokument, welches normalerweise zumindest in Bezug auf das
Erscheinungsbild höchsten 
Standards genügt und insbesondere unabhängig von der Plattform immer gleich aussieht. Auch werden bei
Verwendung der Vorlage, die bereits für das vorliegende Dokument zum Einsatz kam, viele der in
Anhang \ref{anh:Anh-Anforderungen} genannten Anforderungen beinahe automatisch erfüllt. Der anfängliche
Mehraufwand für die Verwendung einer Auszeichnungssprache im Gegensatz zu einem (vermeintlichen)
\gls{wysiwyg}-Programm lohnt sich daher auf jeden Fall.

Wer an den Möglichkeiten von \LaTeX{} Gefallen gefunden hat und mutig ist, der kann dann sogar noch einen
Schritt weiter gehen und auch die Folien für die Abschlusspräsentation mit dem Paket \Paket{latex-beamer}
erstellen und vielleicht später auch für Briefe, Schachaufgaben oder gar den Notensatz von Partituren \LaTeX{} verwenden.
Die Möglichkeiten sind schier endlos.