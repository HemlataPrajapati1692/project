\chapter{Einleitung}
\label{cha:Einleitung}
%
Diese Vorlage basiert auf einer von \textsc{Stefan Macke} auf \gls{github} veröffentlichten Vorlage
\parencite{Macke2009}. Dabei wurde die Struktur weitestgehend übernommen und sinnvoll ergänzt. Die verwendeten
Pakete wurden teilweise aktualisiert. Die wichtigsten dieser Änderungen sind:
%
\begin{compactitem}[\quad$\star$]
\item Ersetzen des veralteten Pakets \texttt{scrpage2}\footnote{%
Die Quellen zu \LaTeX{}-Paketen werden in den seltensten Fällen angegeben, da in den Distributionen normalerweise auch die Dokumentation enthalten ist bzw.~die Dokumente leicht durch eine Internetrecherche auffindbar sind.}
%
  durch das neuere \texttt{scrlayer-scrpage}
\item Verwendung von \texttt{biblatex} mit \texttt{biber} als Backend anstelle von \texttt{bibtex}
\item Erstellung von \emph{Glossar}, \emph{Abkürzungs-} und \emph{Symbolverzeichnis} mit Hilfe des
  modernen \texttt{glossaries}-Pakets statt \texttt{nomencl}
\end{compactitem}
%
Die Details sind den Kommentaren in den entsprechenden Dateien zu entnehmen.

Darüberhinaus wurden die Inhalte der einzelnen Kapitel dem Zweck des Dokuments gemäß angepasst, so dass
vom ursprünglichen Text, der komplett beispielorientiert war, so gut wie gar nichts mehr übrig geblieben ist.

\section{Motivation}
\label{sec:Einl-Motivation}
%
Dieses Dokument dient als Vorlage für Abschlussarbeiten, die mit \LaTeX{} angefertigt werden. Dabei werden
einige Aspekte exemplarisch dargestellt. Ferner wird in Kapitel \ref{cha:Forschungsmethoden} auf einzusetzende
Forschungsmethoden eingegangen, die unabhängig vom verwendeten Textverarbeitungs- und Satzprogramm sind. Auch
die allgemeinen Hinweise zum Zitieren in Kapitel \ref{cha:ZitateReferenzen} sind unabhängig vom
Textverarbeitungsprogramm umzusetzen.

Hinsichtlich \LaTeX{} beginnen wir hier mit den allerersten Aspekten. Man sieht etwa sofort den Unterschied
zwischen z.B. und \zB wenn man auf den Leerraum achtet (dabei wird für die zweite Variante ein eigener Befehl
definiert).
%
\todo{Hier fehlt noch was.}
%
Auch interessant ist sicherlich die hier gezeigte Verwendung von \emph{ToDos}, gerade während des Schreibens der Arbeit. Diese sind ganz am Schluss des Dokuments noch einmal zusammengefasst aufgelistet.


\section{Ziel der Arbeit}
\label{sec:Einl-Ziel}
%
Die Ziele des Dokuments (welches selbst natürlich keine Abschlussarbeit ist \smiley) sind dreierlei:
%
\begin{enumerate}
\item Zunächst soll veranschaulicht werden, welch eindrucksvolles Schriftbild die Dokumentenerstellung mit
  \LaTeX{} hervorbringt und wie gut Aufgaben wie die Erstellung von Verzeichnissen (einschließlich Symbol-
  und Abkürzungsverzeichnissen) automatisiert werden können.
\item Darüber hinaus ist das Dokument so strukturiert, wie es häufig auch für eine Arbeit Sinn ergibt.
  Insofern kann es auch diesbezüglich als Orientierung dienen.
\item Ferner soll das Dokument auch exemplarisch die Umsetzung einiger Anforderungen, etwa zum Zitieren,
  aufzeigen sowie geeignete Forschungsmethoden nennen, die bei der Erstellung einer Abschlussarbeit einzusetzen
  sind.
\end{enumerate}
%
Allerdings werden für die eigene Abschlussarbeit immer auch spezifische Anpassungen etwa mit eigenen
\LaTeX{}-Befehlen oder der Verwendung
spezieller \LaTeX{}-Pakete nötig sein, ebenso wie strukturelle Anpassungen gemäß der eigenen Aufgabenstellung.
Allgemein wird \textbf{erwartet}, dass die jeweiligen Autoren die üblichen Regeln für wissenschaftliches
Arbeiten beachten und insbesondere die \emph{Struktur} der Arbeit der einer wissenschaftlichen Arbeit
entspricht!

\section{Aufbau der Arbeit}
\label{sec:Einl-Aufbau}
%
Das vorliegende Kapitel \ref{cha:Einleitung} bildet die \emph{Einleitung}, in der in das Thema eingeführt
wird und die Ziele abgesteckt werden.
Der \emph{Hauptteil} gliedert sich in drei Kapitel. Zunächst
werden in Kapitel \ref{cha:Forschungsmethoden} Forschungsmethoden vorgestellt, die insbesondere bei
Abschlussarbeiten an \glspl{haw} typischer Weise anwendbar sind. Kapitel \ref{cha:ZitateReferenzen} geht
dann auf das Zitieren ein. Diese beiden Kapitel sind unabhängig vom verwendeten Textverarbeitungsprogramm
relevant. Dahingegen geht  Kapitel \ref{cha:Umgang} darauf ein, wie man wichtige Elemente einer Arbeit mit
\LaTeX{} umsetzen kann. Zum \emph{Schluss} wird in Kapitel \ref{cha:Fazit} ein Fazit gezogen und eine kritische
Bewertung vorgenommen.

Ergänzende Aspekte finden sich im Anhang.

\section{Voraussetzungen zum Verständnis der Arbeit}
\label{sec:Einl-Voraussetzungen}
%
Es bietet sich an, zur Einführung in \LaTeX{} die \LaTeXe{}-Kurzbeschreibung von \textcite{Daniel2018} 
durchzulesen, die bereits die wesentlichen Informationen enthält. Eine ebenfalls kurz gehaltene Einführung
gibt \textcite{Gitter2018}.

\section{Typographische Konventionen}
\label{sec:Einl-Typographie}
%
Namen von Autoren werden in \textsc{Kapitälchen} gesetzt. \emph{Betonung} erfolgt normalerweise durch kursive
Schrift, in besonderen Fällen auch durch \textbf{fette Schrift}. Für Code, Dateinamen sowie die Namen von
\LaTeX{}-Paketen kommt \texttt{Schreibmaschinenschrift} zum Einsatz.

Nachdem in diesem Kapitel die Aufgabenstellung und die Zielsetzung grob abgesteckt wurden, widmet sich das
folgende Kapitel Forschungsmethoden, die typischerweise bei der Erstellung von Abschlussarbeiten an
\gls{haw} anzuwenden sind.