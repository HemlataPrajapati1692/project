%!TEX root = ../../main.tex

\chapter{Introduction}

The project is about predict the outcome of soccer games, training a neuronal network. For the training the database from Kaggl in source \cite{kggl:2019} has been used. 

\section{Motivation}
As master students in informatics it is very interesting to gather knowledge about machine learning and neuronal networks. For this matter a prediction of soccer matches is a very nice procedure to do. Especially because there is a free database with data of matches for a couple of years. In the beginning the knowledge about machine learning is very low and the goal is to improve the handling of python in combination with machine learning techniques. This includes preprocessing data with Pandas, normalize Data Sets and extract the right features, as well as develop adequate models for the learning Algorithm. Additional to the gaining of knowledge we like to create a algorithm, which predict the outcome of soccer games with a decent accuracy.
\section{Goals}
The main goals of this project can be cut into 2 parts:
\begin{itemize}
	\item \textbf{Gaining knowledge}
		\begin{itemize}
			\item Improve skills in Python
			\item Gaining knowledge about data preprocessing 
			\item Gaining knowledge about neuronal networks
		\end{itemize}
	\item \textbf{Outcome of the project}
		\begin{itemize}
			\item Finding the right features \newline
			As a first step it should be done a feature selection. For this the database has to be analyzed and there should be choose some first features by gut feeling. This features has to be checked, if they are independent of each other. During the project it should be always reconsidered, if it make sense to add some more features and checked whether it improves the accuracy. 
			\item Normalizing the features in a proper way \newline
			The features has to be normalized before using them for a prediction model. For this procedure it is necessary to find algorithms or write some. The normalization of the data is a major step in the project development.
			\item Finding a good model for the prediction 
			There are different libraries available for machine learning. The recommended library is Tensorflow in combination with Keras. Additional there has to be research for other libraries and approaches. The different models are compared and in the end the best model will be choose. 
			\item Getting a decent accuracy with the prediction \newline
			The accuracy should be higher than 50\% in the end. With more than 50\% you would theoretically always earning money, if you would bet everything, which the model is predicting. The main goal is to improve the accuracy during the whole project. 
		\end{itemize}
\end{itemize}
\section{Procedure}
