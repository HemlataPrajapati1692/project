\section{Formale Anforderungen}
\label{anh:Anh-Anforderungen}
%
In diesem Teil des Anhangs werden die wichtigsten Anforderungen formaler Art an die Abschlussarbeit nochmals
zusammen gefasst. Sie orientieren sich stark an den Richtlinien zur Erstellung einer wissenschaftlichen
Arbeit der Uni Ulm am dortigen  \textcite{InstitutWirtschaftswissenschaften2016}, wandeln diese aber teilweise
auch ab bzw.~ergänzen sie.
%
\subsection{Aufbau}
\label{subsec:Anh-Anforderungen-Aufbau}
%
Der Aufbau der Abschlussarbeit entspricht dem dieser Vorlage. Dabei ist hinsichtlich der einzelnen Teile noch
zu beachten:
%
\begin{itemize}[$\star$]
\item Die \textbf{Titelseite} entspricht in Ihrer Gestaltung der dieses Dokuments und enthält keine sichtbare
  Seitenzahl.
\item Auf die Titelseite folgen die \textbf{Eigenständigkeitserklärung} sowie die \textbf{Zusammenfassung} und
  deren \emph{englische Version}, das \textbf{Abstract}. Dabei gibt die Zusammenfassung die wesentlichen Punkte
  der Arbeit wieder. Dazu gehören \emph{Zielsetzung}, \emph{Problemstellung} und \emph{Forschungsfragen}, aber
  auch die \emph{Ergebnisse} sowie der \emph{Weg dorthin} sollten kurz genannt werden. All dies sollte nicht
  mehr als eine Seite umfassen!
  
  Auch diese Elemente enthalten keine sichtbaren  Seitenzahlen.
\item Das \textbf{Inhaltsverzeichnis} verweist auf die Seitenzahlen der einzelnen Kapitel und Unterkapitel
  sowie auf Abkürzungs-, Abbildungs-, Tabellen- und Literaturverzeichnis. Falls vorhanden, wird ferner auf
  Glossar und Symbolverzeichnis sowie das Verzeichnis der Listings verwiesen. Die Reihenfolge entspricht dabei
  der dieser Vorlage. Im Inhaltsverzeichnis selbst taucht das Kapitel Inhaltsverzeichnis nicht auf.
  
  Das Inhaltsverzeichnis wird mit römisch I nummeriert. Der gesamte Vorspann an Verzeichnissen wird dann
  weiter \emph{groß römisch} nummeriert.
\item In \textbf{Verzeichnissen nummerierter Elemente} (Abbildungen, Tabellen und Listings) werden die Nummern
  erst nach Kapitel eingeteilt und dann fortlaufend nummeriert (siehe etwa Seite \pageref{listoffigures}).
\item Die \textbf{Einleitung} dient der kurzen Präsentation des Themas, angefangen mit der Ausgangssituation
  und Themendarstellung, über die Motivation hin zur Problembeschreibung (inkl.~der konkreten Forschungsfrage(n))
  und deren thematischer Abgrenzung. Zum Abschluss der Einleitung ist der \emph{Aufbau der Arbeit} so
  darzustellen, dass der rote Faden erkennbar ist. Im Unterschied zur \emph{Zusammenfassung} wird in der
  Einleitung nicht auf die Ergebnisse eingegangen.
  
  Beginnend mit der Einleitung hat die Arbeit bis zum Anhang nun normale arabische Seitenzahlen. Am Schluss der
  Einleitung steht eine \emph{Überleitung} zum Hauptteil, die sich am \emph{roten Faden} der Aufgabenstellung
  orientiert.
\item Im \textbf{Hauptteil} bildet das Literaturreview gemäß Abschnitt \ref{sec:FM-Literaturreview} die Basis.
  Die weiteren Kapitel sind dann von der konkreten Aufgabenstellung abhängig. Im Falle einer Aufgabenstellung
  aus dem Bereich \emph{Data Science} eignet sich eine Strukturierung wie in Abschnitt 
  \ref{sec:FM-DataScienceVorgehen} beschrieben.
  
  In jedem Fall ist beim Schreiben auf den \emph{roten Faden} zu achten. Dies bedeutet, dass zum Ende eines Kapitels
  bzw.~beim Übergang auf ein anderes Thema ein kurzes \emph{Zwischenfazit} gezogen sowie auf den folgenden Teil
  vorausgeblickt werden sollte.
\item Der \textbf{Schluss} sollte in erster Linie noch einmal die wichtigsten Ergebnisse der Arbeit prägnant
  zusammenfassen. Ein Ausblick auf weitere Fragestellungen sowie eigene Gedanken sind zusätzlich möglich. Dabei
  ist ein Verweis auf weiterführende Literatur grundsätzlich wünschenswert.
\item Das \textbf{Literaturverzeichnis} ist gemäß den Ausführungen in Kapitel \ref{cha:ZitateReferenzen} zu
  erstellen.
\item Der \textbf{Anhang} enthält ergänzende Informationen zur Abschlussarbeit, die im Hauptteil keinen Platz
  gefunden haben oder dort dem Lesefluss hinderlich gewesen wären. Typischer Weise finden in einem Anhang zusätzliche
  Abbildungen oder Tabellen, aber eventuell auch längere Codeabschnitte Platz (obwohl gerade letztere möglichst
  nur digital eingereicht werden sollten). Wurden Interviews durchgeführt oder Mock-Ups erstellt, so können diese
  im Anhang dokumentiert werden.
  
  Der Anhang als Kapitel wird mit dem Buchstaben A nummeriert und beginnt mit einer Seite, auf der lediglich
  \emph{A. Anhang} steht und die auch keine sichtbare Seitenzahl enthält.
  Sämtliche Teile des Anhangs werden dann als Abschnitte behandelt und mit A.1, A.2, 
  \ldots{} nummeriert. Die Seitenzahlen im Anhang sind \emph{kleine römische Ziffern}.
\end{itemize}
%
\subsection{Äußere Form}
\label{subsec:Anh-Anforderungen-Form}
%
\begin{itemize}[\quad$\star$]
\item Hinsichtlich der \textbf{Sprache} ist sowohl die \emph{deutsche} wie auch die \emph{englische} Sprache für das
  Verfassen der Arbeit erlaubt. Dies ist lediglich vorab zu vereinbaren.
\item Hinsichtlich der \textbf{Formatierung} gelten folgende Regeln:
  \begin{compactitem}[\quad\checkmark]
    \item Die Arbeit ist \emph{doppelseitig} zu drucken.
    \item \emph{Seitenränder}: oben: 20mm, unten: 20mm, innen: 25mm, außen: 25mm, Bindeabstand: 5mm.
    \item \emph{Schriftgröße}: 12-Punkt-Schrift, \emph{Hauptschriftart}: serifenbehaftet und einheitlich im
      gesamten Dokument.
    \item \emph{Zeilenabstand}: 1,5-fach
    \item Es wird \emph{Blocksatz} und \emph{Silbentrennung} erwartet!
    \item Korrekte \emph{Rechtschreibung} und \emph{Zeichensetzung} sind wichtig (\zB 
      \href{http://www.das-dass.de/}{das/dass}).
    \item Neue Kapitel und Abschnitte sollten nicht alleinstehend ohne weiteren Text am unteren Seitenende beginnen!
    \item Hat ein Kapitel Abschnitte oder ein Abschnitt Unterabschnitte, dann müssen es jeweils mindestens zwei
      pro übergeordneter Einheit sein, also etwa
      \begin{center}
        3.1 Programmiersprachen --- 3.1.1 Python --- 3.1.2 Java
      \end{center}
    \item Sätze sind \emph{auszuformulieren} und somit sind stichpunktartige Formulierungen nicht zulässig!
  \end{compactitem}
\end{itemize}
%
\subsection{Verweise}
\label{subsec:Anh-Anforderungen-Verweise}
%
Zu Verweisen, insbesondere zum Zitieren, wurde bereits in Kapitel \ref{cha:ZitateReferenzen} Einiges ausgeführt. 
Die dort bereits explizit genannten \emph{Prinzipien} sind daher zu beachten. Hier werden diese nun wiederholt,
aber auch noch weitere Anforderungen aufgelistet.
%
\begin{compactitem}[\quad$\star$]
\item Hinsichtlich \textbf{Zitaten} gelten auch folgende Regeln:
  \begin{compactitem}[\quad\checkmark]
  \item Hat man etwas aus Quellen übernommen (Text, Ideen, Abbildungen, Code, etc.), so muss man die Quelle auch angeben!
  \item Es ist der Harvard-Stil, auch Autor-Jahr-Zitierweise genannt, mit Kurzbelegen zu verwenden!
  \item Bezieht sich ein Textabschnitt auf die gleiche Quelle, so reicht eine Angabe am Anfang oder Ende des
    jeweiligen Abschnitts. Bei sehr langen Abschnitten sollte jedoch möglichst früh Bezug auf die Quelle genommen werden.
  \item Es ist nicht zulässig beim (erstmaligen) Zitieren, einfach nur den Namen eines Autors im Text zu nennen.
    \begin{compactitem}[\quad--]
    \item falsch: \glqq Laut \textsc{Gitter} lohnt sich die Beschäftigung mit \LaTeX{}.\grqq
    \item richtig: \glqq Laut \textcite{Gitter2018} lohnt sich die Beschäftigung mit \LaTeX{}.\grqq
    \end{compactitem}
  \item Wenn, und nur wenn, eindeutig klar ist, auf welche Literatur sich eine Aussage bezieht, \dh wenn ein ganzer
    Abschnitt lediglich eine Quelle behandelt, kann auf die wiederholte Nennung der Jahreszahl im Zitat verzichtet 
    oder auf die Formulierung der \glqq Autor/die Autorin\grqq{} zurückgegriffen werden.
  \item Grundwissen aus Lehrbüchern muss nicht explizit mit einer Quellenangabe versehen werden.
  \item Nutzen Sie als Quellen soweit möglich seriöse Literatur. Für die Nutzung von Internetquellen muss es einen
    guten Grund geben (der normalerweise dann auch zu nennen ist).
  \item Ein Zitat im Text enthält weder den Vornamen des Autors/der Autorin, noch den Titel der Arbeit, noch den Verlag.
    Im Literaturverzeichnis sind diese Angaben jedoch zwingend erforderlich. Bei mehr als zwei Autoren, ist die Angabe 
    im Text entsprechend mit \glqq et al.\grqq{} oder \glqq u.\,a.\grqq{} abzukürzen \parencite[vgl~][]{Chapman1999}.
  \end{compactitem}
\item Hinsichtlich \textbf{Abbildungen und Tabellen} ist noch zu beachten:
  \begin{compactitem}[\quad\checkmark]
  \item Die \emph{Quelle} ist direkt in der Bild- bzw.~Tabellenunterschrift anzugeben.
    \begin{compactitem}[\quad--]
    \item Bei selbst erstellten Abbildungen und Tabellen findet sich als Quelle: \glqq eigene Darstellung\grqq{} bzw. 
      \glqq eigene Berechnungen\grqq.
    \item Wurde eine Graphik aus einer Quelle rekonstruiert, so ist die Quelle mit anzugeben, etwa: 
      \glqq\parencite[eigene Darstellung nach][]{Mayring1994}\grqq.
    \end{compactitem}
  \item Für Abbildungen, Tabellen und Listings ist immer auch ein \emph{Verweis im Fließtext} notwendig, \dh die
    genannten Objekte dürfen nicht ohne Bezug im Text auftreten.
  \end{compactitem}
\end{compactitem}
%
\subsection{Sonstige Anforderungen}
\label{subsec:Anh-Anforderungen-Sonstige}
%
\begin{itemize}[\quad$\star$]
\item \emph{Tabellen und Abbildungen} sind zu zentrieren!
\item \emph{Rekonstruktionen} von Tabellen und Abbildungen sind immer Screenshots oder Scans der Originale
  vorzuziehen. In jedem Fall ist auf eine gute \emph{Darstellungsqualität} zu achten!
\item Im \emph{Literaturverzeichnis}
    \begin{compactitem}[\quad--]
    \item sind Onlinequellen von wissenschaftlichen Quellen wie Büchern und Artikeln in
      (wissenschaftlichen) Publikationen zu trennen!
    \item dürfen keine Blindquellen auftauchen und daher sind Rückverweise zu verwenden!
    \item sind normalerweise weder \gls{doi} noch URLs anzugeben, wobei letztere bei Onlinequellen natürlich
      benötigt werden!
    \end{compactitem}
\item Bei der ersten Verwendung einer \emph{Abkürzung} / eines \emph{Akronyms} im Text ist der Begriff
  zusätzlich auszuschreiben bzw. zu erklären!
\end{itemize}

Besondere Anforderungen werden an die Abgabe der Arbeit gestellt. Diese sind daher im nachfolgenden Abschnitt des Anhangs gesondert zusammengefasst.