\section{Aufzählungen}
\label{sec:Aufzaehlungen}
%
Aufzählungen verschiedener Art treten natürlich häufig in Schriftstücken auf und sollen daher auch speziell
erwähnt werden. In diesem Zusammenhang bietet das Paket \Paket{paralist} erweiterte
Möglichkeiten. Es folgen einige Beispiele.
%

Zunächst eine normale Punktliste:
\begin{itemize}
\item Hat erst einen Punkt (bzw.~Spiegelstrich)
\item und dann noch einen zweiten
\item sowie beliebig viele weitere
\end{itemize}

Oder lieber wirklich mit Strichen statt Punkten (einfach änderbar durch \Paket{paralist}):
%
% der (optionale) Parameter funktioniert wegen des paralist-Pakets
\begin{itemize}[\quad---]
\item Hat erst einen Punkt (bzw.~Spiegelstrich)
\item und dann noch einen zweiten
\item sowie beliebig viele weitere
\end{itemize}

Oder obige Listen mit \Code{compactitem} aus dem \Paket{paralist}-Paket:
\begin{compactitem}
\item Hat erst einen Punkt (bzw.~Spiegelstrich)
\item und dann noch einen zweiten
\item sowie beliebig viele weitere
\end{compactitem}

\begin{compactitem}[\quad---]
\item Hat erst einen Punkt (bzw.~Spiegelstrich)
\item und dann noch einen zweiten
\item sowie beliebig viele weitere
\end{compactitem}


Unterlisten sind selbstverständlich auch möglich:
\begin{itemize}
\item Hat erst einen Punkt (bzw.~Spiegelstrich)
	\begin{compactitem}
	\item und einen Unterpunkt
	\item und dann noch einen zweiten
	\end{compactitem}
\item und dann noch einen zweiten
\item sowie beliebig viele weitere
\end{itemize}

Eine nummerierte Liste hingegen sieht so aus:
\begin{enumerate}
\item Es geht um dies und
\item um das.
\end{enumerate}

Eine kompakte nummerierte Liste mit Kleinbuchstaben statt Zahlen erscheint folgendermaßen:
\begin{compactenum}[\quad(a)]
\item Es geht um dies und
\item um das.
\end{compactenum}

Beschreibungen sind auch möglich (hier wieder in kompakter Form):
\begin{compactdesc}
\item[\quad Hund] ein Kläffer
\item[\quad Katze] eine Schnurrerin
\item[\quad Maus] fehlt noch für die Montagsmaler
\end{compactdesc}
%
Nach diesen Eindrücken verschiedener Aufzählungen soll es im letzten Abschnitt des Kapitels noch um
Software zur Erstellung der \LaTeX{}-Dokumente sowie für die Literaturverwaltung gehen.
%
\todo[size=\footnotesize, color=green!30, noline]{Überleitung\checkmark}
