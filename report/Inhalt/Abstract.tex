\section*{Zusammenfassung}
\label{sec:Zusammenfassung}
Diese Vorlage dient der Erstellung von Abschlussarbeiten mit \LaTeX.
Dabei werden einige Aspekte direkt angesprochen, während andere sich erst erschließen, indem man sich die Verzeichnis- und Dateistruktur sowie die Inhalte der eingebundenen Dateien ansieht. Aus den Kommentaren wird klar, was der jeweilige Zweck ist. Ansonsten hilft eine kurze Internetrecherche meist weiter.

Einige Kapitel, insbesondere diejenigen zu den Forschungsmethoden (\ref{cha:Forschungsmethoden}) und zum Zitieren
(\ref{cha:ZitateReferenzen}), sind jedoch auch dann zu beachten, wenn die Erstellung der Arbeit mit einem anderen
Textverarbeitungs- und Satzprogramm erfolgt. Dies gilt natürlich auch für Anhang \ref{anh:Anh-Anforderungen}!

Das Dokument zeigt, dass mit \LaTeX{} optisch äußerst hochwertige Dokumente erzeugt werden können, bei denen
sich das Programm außerdem noch automatisch um die Einhaltung weiterer Anforderungen kümmert.


\section*{Abstract}
\label{sec:Abstract}
This template shall help in writing a thesis using \LaTeX.
Some issues are directly addressed, while others only become clear once the structure of folders and files is inspected as well as the contents of the files themselves. From the comments in the files their respective purpose should be apparent. If not, a brief search on the internet should help solve any remaining issues.

Some chapters, especially those on research methods (\ref{cha:Forschungsmethoden}) and on citations
(\ref{cha:ZitateReferenzen}), however, are also relevant when writing the thesis using some other text
processor. This is obviously also true for appendix \ref{anh:Anh-Anforderungen}!

The document shows, that using \LaTeX{} will produce optically pleasing results, while at the same time taking
care of fulfilling additional requirements automatically.