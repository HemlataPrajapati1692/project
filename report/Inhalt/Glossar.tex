% siehe https://www.dante.de/events/dante2015/Programm/vortraege/vortrage-partosch.pdf für weitere Informationen
% Abkürzungen

\newacronym{api}{API}{Application Programming Interface}
\newacronym{aris}{ARIS}{Architektur integrierter Informationssysteme}
\newacronym{bpr}{BPR}{Business Process Reengineering}
\newacronym{eepk}{eEPK}{erweiterte Ereignisgesteuerte Prozesskette}
\newacronym{epk}{EPK}{Ereignisgesteuerte Prozesskette}
\newacronym{jms}{JMS}{Java Message Service}
\newacronym{sdk}{SDK}{Software Development Kit}
\newacronym{uri}{URI}{Uniform Resource Identifier}
\newacronym{url}{URL}{Uniform Resource Locator}
\newacronym{urn}{URN}{Uniform Resource Name}
\newacronym{w3c}{W3C}{World Wide Web Consortium}
\newacronym{xml}{XML}{Extensible Markup Language}
\newacronym{xpath}{XPath}{XML Path Language}
\newacronym{xsl}{XSL}{Extensible Stylesheet Language}
\newacronym{xslt}{XSLT}{XSL Transformations}

\newacronym{ide}{IDE}{Integrated Development Environment (dt.: Integrierte Entwicklungsumgebung)}
\newacronym{wysiwyg}{WYSIWYG}{What You See Is What You Get}
\newacronym{degeval}{DeGEval}{Deutsche Gesellschaft für Evaluation}
\newacronym[plural={FAQ}, longplural={frequently asked questions %
  (dt.: häufig gestellte Fragen)}]{faq}{FAQ}{frequently asked question %
  (dt.: häufig gestellte Frage)}
\newacronym{mcda}{MCDA}{Multi Criteria Decision Analysis %
   (dt.: multikriterielle Entscheidungsanalyse)}
\newacronym[plural={HAWn}, longplural={Hochschulen für Angewandte %
  Wissenschaften}]{haw}{HAW}{Hochschule für Angewandte Wissenschaften}

% Glossareinträge

\newglossaryentry{crispdm}{name=\textsf{Cross-Industry Standard Process for Data Mining (CRISP-DM)}, %
description={bereits 1999 in einer ersten Version veröffentlichtes Vor\-ge\-hens\-mo\-dell für Data Mining,  welches 
  auch heute noch häufig bei Data Science Aufgaben angewandt wird}}
  
\newglossaryentry{dante}{name=\textsf{Deutschsprachige Anwendervereinigung TeX e.V.}, %
description={1989 in Heidelberg gegründeter Verein mit dem Zweck der Betreuung und %
  Beratung von \TeX{}-Benutzern im gesamten deutschsprachigen Raum}}
  
\newglossaryentry{doi}{name=\textsf{Digital Object Identifier}, %
description={eindeutiger und dauerhafter digitaler Identifikator für physische, digitale % 
  oder abstrakte Objekte}}
  
\newglossaryentry{github}{name=\textsf{GitHub}, %
description={Plattform für die kollaborative Bearbeitung von Projekten (insbes.~Entwicklung von Software) %
  mit Versionskontrolle}}
  
\newglossaryentry{plagiat}{name=\textsf{Plagiat}, %
description={unrechtmäßige Aneignung von Gedanken, Ideen o.\,Ä.~eines anderen auf künstlerischem oder wissenschaftlichem %
  Gebiet und ihre Veröffentlichung; Diebstahl geistigen Eigentums; durch eine der zuvor genannten Verfehlungen entstandenes Werk}}
  
% Eintraege für Symbolliste

\newglossaryentry{kappa}{type=symbols, name={\ensuremath{\kappa}}, %
sort=kappa, symbol={\ensuremath{\kappa}}, %
description={Anzahl verschiedener Klassen}}

\newglossaryentry{datenbasis}{type=symbols, name={\ensuremath{\mathcal{D}}}, %
sort=datenbasis, symbol={\ensuremath{\mathcal{D}}}, %
description={Datenbasis mit den Datensätzen $d_1,d_2,\dots,d_m$}}%